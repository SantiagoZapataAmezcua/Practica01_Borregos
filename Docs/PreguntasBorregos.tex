\documentclass[12pt]{report}
\usepackage[utf8]{inputenc}
\usepackage[T1]{fontenc}
\usepackage[spanish,es-noshorthands]{babel}
\usepackage[margin=1in]{geometry}
\usepackage{array}
\usepackage{amsmath, amssymb, mathrsfs}
\usepackage{fdsymbol}
\usepackage{forest}
\usepackage{prooftrees}
\usepackage{tabularx}
\usepackage{graphicx}
\usepackage{float}
\usepackage{booktabs}
\usepackage{marginnote}
\usepackage{tcolorbox}
\usepackage{tikz}
\usetikzlibrary{shapes.geometric, arrows}
\usepackage{xcolor}
\usepackage{enumitem}
\usepackage{hyperref}

\definecolor{azul}{RGB}{25,94,164}
\setlist[itemize]{leftmargin=*, topsep=4pt}
\setlist[enumerate]{leftmargin=*, topsep=6pt}

\begin{document}

% ------------------ PORTADA (tal cual tu formato) ------------------
\pagenumbering{gobble}
\begin{titlepage}
\centering
{\bfseries\LARGE Universidad Nacional Autonoma de M\'exico \par}
\vspace{1cm}
{\scshape\Large Facultad de Ciencias \par}
\vspace{3cm}
{\scshape\Huge Practica 1  \par}
\vspace{3cm}
{\itshape\Large Fundamentos de Bases de Datos \par}
\vfill
{\scshape\Huge Equipo "BORREGOS"  \par}
\vspace{3cm}
{\Large Aron\\ Rocks\\ Isac\\ Kevin\\ Santiago Zapata Amezcua \par}
\vfill
\end{titlepage}
\clearpage

% ------------------ CONTENIDO (mismo formato que antes) -----------
\pagenumbering{arabic}
\setcounter{page}{1}

{\LARGE\bfseries\textcolor{azul}{Actividades.}}\par\vspace{0.5em}

En la sesión de laboratorio realizaremos la instalación de \textbf{Docker} y del \textbf{SMBD PostgreSQL} y explicaremos de manera poco detallada los componentes en pantalla de \textbf{DBeaver}.

\bigskip

\begin{enumerate}[label=\roman*.]

\item \textbf{Se debe de responder las siguientes preguntas en equipo} y se entrega en un documento \textbf{PDF} con el nombre \textbf{Preguntas} seguido del \textbf{nombre} que le hayan dado a su equipo.

\medskip

\textbf{Preguntas:}

\begin{enumerate}[label=\textbf{\arabic*.}, itemsep=2.0em]
  \item ¿Qué otros \textbf{SMBD} existen actualmente en el mercado?\\
    \textbf{Solucion:}\\
    \textbf{1. Relacionales (SQL tradicionales)}\\
Se basan en el modelo relacional propuesto por Edgar F. Codd, usan tablas con filas y columnas, ademas de usar SQL como lenguaje estándar y son ideales cuando los datos son estructurados y requieren consistencia.\\

Oracle Database
\begin{itemize}
    \item Muy usado en corporativos grandes.
    \item Altamente robusto, escalable y con herramientas de seguridad.
    \item Costoso en licenciamiento.
\end{itemize}

Microsoft SQL Server
\begin{itemize}
    \item Popular en empresas que ya usan tecnologías Microsoft.
    \item Buena integración con herramientas de análisis (Power BI, Excel, Azure).
    \item Tiene versiones de pago y una gratuita (SQL Server Express).
\end{itemize}

MariaDB
\begin{itemize}
    \item Derivado de MySQL, creado tras la adquisición de MySQL por Oracle.
    \item Es totalmente libre y de código abierto.
    \item Compatible con MySQL pero con mejoras en rendimiento y nuevas funciones.
\end{itemize}

IBM Db2
\begin{itemize}
    \item Orientado a entornos empresariales con grandes volúmenes de datos.
    \item Usado en sectores financieros y gubernamentales.
\end{itemize}
\newpage
\textbf{2. No Relacionales (NoSQL)}\\
Diseñados para manejar datos no estructurados o semiestructurados, como documentos, grafos o claves-valor.\\

MongoDB
\begin{itemize}
    \item Basado en documentos JSON.
    \item Muy flexible y escalable.
    \item Ideal para aplicaciones web y big data.
\end{itemize}

Cassandra (Apache Cassandra)
\begin{itemize}
    \item Basado en un modelo distribuido.
    \item Excelente para grandes volúmenes de datos con alta disponibilidad.
    \item Usado por empresas como Netflix y Facebook.
\end{itemize}

Redis
\begin{itemize}
    \item Base de datos en memoria.
    \item Extremadamente rápida, usada para caché y operaciones en tiempo real.
\end{itemize}

Neo4j
\begin{itemize}
    \item Especializada en bases de datos de grafos.
    \item Ideal para aplicaciones de redes sociales, recomendaciones o detección de fraudes.
\end{itemize}

\textbf{3. En la nube (Database as a Service, DBaaS)}\\
Cada vez más comunes, gestionado por un proveedor en la nube. Ofrecen escalabilidad automática y administración simplificada.\\

\begin{itemize}
    \item Amazon RDS (Relational Database Service) $\rightarrow$ soporta PostgreSQL, MySQL, MariaDB, Oracle y SQL Server.
    \item Google Cloud SQL $\rightarrow$ permite desplegar y utilizar PostgreSQL y MySQL.
    \item Azure SQL Database $\rightarrow$ es la versión en la nube de Microsoft SQL Server.
    \item Firestore / DynamoDB (de Google y AWS) $\rightarrow$ NoSQL altamente escalables en la nube.
\end{itemize}
  \item ¿Cuáles son las principales diferencias con \textbf{PostgreSQL}?

  \item ¿Por qué una empresa debería escoger una \textbf{base de datos open source}?

  \item ¿Cuáles son las ventajas, para un \textbf{DBA}, de trabajar con un \textbf{SMBD open source}?

  \item ¿Qué son las \textbf{bases de datos NoSQL}? Menciona \textbf{3 ventajas y 3 desventajas} contra las bases relacionales.\\
  Las \textbf{bases de datos NoSQL} son bases de datos no relacionales que almacenan los datos de forma diferente a las bases de datos 
  relacionales, es decir, las bases de datos NoSQL no almacenan los datos en tablas relacionales basadas en reglas, de ahí que algunas 
  personas dicen que el término NoSQL se refiere a "no SQL" o también "no solo SQL". El modelo de esquema de las bases de datos NoSQL es 
  flexible y permiten los datos no estructurados como por ejemplo documentos, pares clave-valor, columnas o gráficas.\\
  \textbf{Ventajas} de las \textit{bases de datos NoSQL}:
  \begin{itemize}
      \item \textbf{Escalabilidad:\\} 
      Las bases de datos NoSQL son capaces de poder aumentar la capacidad conforme los datos y el tráfico crecen, además tienen funciones 
      que ayudan al ajuste de escala automático.
      \item \textbf{Consultas más rápidas:\\} 
      Las bases de datos NoSQL están optimizadas para consultas rápidas, además por lo general no usan uniones complejas en las búsquedas 
      y por lo tanto devuelven los resultados de manera más rápida y estas no necesitan normalización como las bases de datos relacionales.
      \item \textbf{Desarrollo ágil:\\} 
      Gracias a la flexibilidad con la que cuentan las bases de datos NoSQL los desarrolladores no se tienen que preocupar mucho en la 
      transformación de datos y pueden iterar sobre ellos más rápidamente ya que la base de datos NoSQL almacena varios tipos de datos aún 
      en su formato original y que el modelo se adapte conforme lo ocupes y así desarrollar aplicaciones de manera más eficiente.
  \end{itemize}
  \textbf{Desventajas} de las \textit{bases de datos NoSQL}:
  \begin{itemize}
      \item \textbf{Madurez:\\} 
      Dado que las bases de datos NoSQL son relativamente nuevas a diferencia de las bases de datos relacionales, pueden carecer de madurez, 
      además en general los desarrolladores tienen menos experiencia al trabajar con estas bases de datos, o pueden verse afectados por no 
      encontrar suficiente documentación o ayuda para ciertos problemas.
      \item \textbf{Lenguaje:\\} 
      Las bases de datos NoSQL no tienen un lenguaje estándar como las relacionales que es SQL, entonces, cada base de datos puede tener su 
      propio lenguaje para la administración y consultas de los datos.
      \item \textbf{Limitaciones:\\} 
      En algunos casos las bases de datos NoSQL no garantizan la protección de integridad de los datos ni un nivel alto en la consistencia 
      de los datos a como lo hacen las bases de datos relacionales.
  \end{itemize}
\end{enumerate}

\bigskip




\item \textbf{Cada integrante} deberá realizar un \textbf{reporte} en donde indiquen todos los pasos que realizaron para la instalación de \textbf{Docker} y de \textbf{PostgreSQL}. Además, deberán incluir \textbf{cómo se conectaron con DBeaver}, que es la herramienta que utilizaremos durante el curso. 

Cada bitácora debe ser un documento \textbf{PDF} que inicie con la palabra \textbf{Bitácora} seguida del \textbf{apellido paterno} de la persona que elaboró su documento.

\medskip

\textbf{Los puntos que se requieren en su reporte de instalación son:}

\begin{enumerate}[label=\textbf{\arabic*.}, leftmargin=*, itemsep=0.8em]
  \item \textbf{Sistema operativo y versión.}
  \item \textbf{Distribución} (solamente en el caso de \textit{Linux}).
  \item \textbf{Versión de la instalación.}
  \item \textbf{Tiempo requerido.}
  \item \textbf{Explicación del paso a paso} que realizaste con sus respectivas capturas de pantalla. (Adicionalmente agrega las evidencias de los pasos que consideres esenciales de la instalación).
  \item \textbf{Comentarios y problemas} a los que te enfrentaste en la instalación.
\end{enumerate}

\end{enumerate}

\begin{thebibliography}{99}
    \bibitem{inesem}
        Marín, R. (2024, 30 diciembre). Los gestores de bases de datos más usados en la actualidad. Canal Informática y TICS. https://www.inesem.es/revistadigital/informatica-y-tics/los-gestores-de-bases-de-datos-mas-usados
    \bibitem{wbadmin}
        Wbadmin. (2025, 2 marzo). Las 10 Bases de Datos más Populares – Febrero 2025. Productos y Servicios Digitales. https://nubecolectiva.com/blog/las-10-bases-de-datos-mas-populares-febrero-2025
    \bibitem{Gemini}
        Google Gemini. (s. f.). Gemini. https://gemini.google.com/app?hl=es
    \bibitem{Cloud}
        What is NoSQL? Databases Explained | Google Cloud. (s. f.-b). Google Cloud. https://cloud.google.com/discover/what-is-nosql

\end{thebibliography}

\end{document}
