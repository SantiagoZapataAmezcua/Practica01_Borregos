\documentclass[12pt]{report}
\usepackage[utf8]{inputenc}
\usepackage[T1]{fontenc}
\usepackage[spanish,es-noshorthands]{babel}
\usepackage[margin=1in]{geometry}
\usepackage{array}
\usepackage{amsmath, amssymb, mathrsfs}
\usepackage{fdsymbol}
\usepackage{forest}
\usepackage{prooftrees}
\usepackage{tabularx}
\usepackage{graphicx}
\usepackage{float}
\usepackage{booktabs}
\usepackage{marginnote}
\usepackage{tcolorbox}
\usepackage{tikz}
\usetikzlibrary{shapes.geometric, arrows}
\usepackage{xcolor}
\usepackage{enumitem}
\usepackage{hyperref}

\definecolor{azul}{RGB}{25,94,164}
\setlist[itemize]{leftmargin=*, topsep=4pt}
\setlist[enumerate]{leftmargin=*, topsep=6pt}

\begin{document}

% ------------------ PORTADA (tal cual tu formato) ------------------
\pagenumbering{gobble}
\begin{titlepage}
\centering
{\bfseries\LARGE Universidad Nacional Autonoma de M\'exico \par}
\vspace{1cm}
{\scshape\Large Facultad de Ciencias \par}
\vspace{3cm}
{\scshape\Huge Practica 1  \par}
\vspace{3cm}
{\itshape\Large Fundamentos de Bases de Datos \par}
\vfill
{\scshape\Huge Equipo "BORREGOS"  \par}
\vspace{3cm}
{\Large Aron\\ Rocks\\ Isac\\ Kevin\\ Santiago Zapata Amezcua \par}
\vfill
\end{titlepage}
\clearpage

% ------------------ CONTENIDO (mismo formato que antes) -----------
\pagenumbering{arabic}
\setcounter{page}{1}

{\LARGE\bfseries\textcolor{azul}{Actividades.}}\par\vspace{0.5em}

En la sesión de laboratorio realizaremos la instalación de \textbf{Docker} y del \textbf{SMBD PostgreSQL} y explicaremos de manera poco detallada los componentes en pantalla de \textbf{DBeaver}.

\bigskip

\begin{enumerate}[label=\roman*.]

\item \textbf{Se debe de responder las siguientes preguntas en equipo} y se entrega en un documento \textbf{PDF} con el nombre \textbf{Preguntas} seguido del \textbf{nombre} que le hayan dado a su equipo.

\medskip

\textbf{Preguntas:}

\begin{enumerate}[label=\textbf{\arabic*.}, itemsep=2.0em]
  \item ¿Qué otros \textbf{SMBD} existen actualmente en el mercado?

  \item ¿Cuáles son las principales diferencias con \textbf{PostgreSQL}?

  \item ¿Por qué una empresa debería escoger una \textbf{base de datos open source}?

  \item ¿Cuáles son las ventajas, para un \textbf{DBA}, de trabajar con un \textbf{SMBD open source}?

  \item ¿Qué son las \textbf{bases de datos NoSQL}? Menciona \textbf{3 ventajas y 3 desventajas} contra las bases relacionales.
\end{enumerate}

\bigskip

\item \textbf{Cada integrante} deberá realizar un \textbf{reporte} en donde indiquen todos los pasos que realizaron para la instalación de \textbf{Docker} y de \textbf{PostgreSQL}. Además, deberán incluir \textbf{cómo se conectaron con DBeaver}, que es la herramienta que utilizaremos durante el curso. 

Cada bitácora debe ser un documento \textbf{PDF} que inicie con la palabra \textbf{Bitácora} seguida del \textbf{apellido paterno} de la persona que elaboró su documento.

\medskip

\textbf{Los puntos que se requieren en su reporte de instalación son:}

\begin{enumerate}[label=\textbf{\arabic*.}, leftmargin=*, itemsep=0.8em]
  \item \textbf{Sistema operativo y versión.}
  \item \textbf{Distribución} (solamente en el caso de \textit{Linux}).
  \item \textbf{Versión de la instalación.}
  \item \textbf{Tiempo requerido.}
  \item \textbf{Explicación del paso a paso} que realizaste con sus respectivas capturas de pantalla. (Adicionalmente agrega las evidencias de los pasos que consideres esenciales de la instalación).
  \item \textbf{Comentarios y problemas} a los que te enfrentaste en la instalación.
\end{enumerate}

\end{enumerate}

\end{document}
